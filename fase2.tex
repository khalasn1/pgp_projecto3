%formatação do documento e tipo do documento
\documentclass[12pt, a4paper, twoside]{report} %inicio do doc

%pacotes de extensões
\usepackage[portuges]{babel} %pkg da lingua portugues
\usepackage[latin1,utf8]{inputenc} %pkg da lingua portugues
\usepackage{verbatim} %pkg para escrever sem formataçao
\usepackage{color} %usar cores nas letras
\usepackage{graphicx} %usar imagens no doc
\usepackage[table,xcdraw]{xcolor}
\usepackage{makeidx}
\usepackage{anysize} % para formatar o tamanho do documento
\usepackage{multirow}


\marginsize{3.17cm}{3.17cm}{2.54cm}{2.54cm}

%fazer índice
\makeindex

\begin{document}

\title{%
	\textbf{Planeamento e Gestão de Projecto}\\ 
	\large Relatório Fase 2
}

\author{%
Alexandre Machado, nº 43551 \\
Nuno Silva, nº 44285 \\
Francisco Pires, nº 44314 \\
}

\date{\today}
\maketitle
\tableofcontents

\chapter{Introdução}

\textit{Foram feitas alterações nas partes entregues na primeira fase.} \\


Este projecto tem como objectivo o desenvolvimento e a implementação de um Sistema de Informação (SI), dirigido aos utentes do Serviço Nacional de Saúde (SNS). 
Este SI é baseado em tecnologias \textit{web} e pretende melhorar a qualidade dos serviços prestados ao utilizador. 
Após a consulta do Portal da Saúde, e a identificação das capacidades existentes, propomos ampliar os requisitos funcionais disponíveis para o utilizador e melhorar os requisitos não funcionais. 
Para isso, pretendemos assegurar a melhor disponibilidade dos servidores, correcções na interface do \textit{website} e confidencialidade dos dados associados ao utilizador.

\chapter{Análise de requisitos}

\section{Requisitos funcionais e não funcionais}

\subsection{Requisitos funcionais}

De acordo com os objectivos definidos para este projecto, selecionamos as seguintes funcionalidades que o utilizador terá disponíveis neste SI.

\begin{itemize}

\item Registo de contactos e dados pessoais
\item Definir agregado familiar 
\item Identificação de cuidador familiar
\item Registo de informação pessoal relevante
\item Registo de indicadores básicos de saúde
\item Registo de exames complementares de diagnóstico
\item Consulta de registos clínicos
\item Pedido de prescrição de medicação crónica
\item Marcação de consultas
\item Inscrição e consulta das listas para cirurgia (eSIGIC)
\item Testamento vital
\item Definir estado no Registo Nacional de Não Dadores (RENNDA)
\item Pesquisa de serviços médicos (directório)
\item Pedido de mudança de médico de família
\item Pedido de isenção de taxas moderadoras

\end{itemize}

\subsection{Requisitos não funcionais}

Para execução das funcionalidades neste SI, será necessário assegurar os requisitos não funcionais que listamos de seguida.

\begin{itemize}
\item Confidencialidade dos dados
\item Segurança dos dados e dos acessos
\item Garantia de disponibilidad
\item Escalável e modular
\item Tempo de resposta
\item Assegurar o cumprimentos das normas legais
\item Resolução de conflitos
\item Persistência e sincronização dos dados
\item Notificações e alertas de acontecimentos do utilizador
\item \textit {Responsive Web Design}
\end{itemize}

\chapter{Planeamento}

\section{Recursos}

\textbf{Recursos Humanos}
\\

Os recursos humanos para o projecto incluem seis alunos de Tecnologias de Informação (LTI), sendo os restantes alunos são do grupo 003.
Ao fim da cadeira de Planeamento e Gestão do Projecto (PGP), os dois grupos vão-se juntar e trabalhar em conjunto nas cadeiras de Projecto Tecnologias de Informação (PTI) e Projecto Tecnologias de Redes (PTR). A duração do projecto vai ser de três meses e meio, mais 3 meses e meio de planeamento. 
\\
\\
\textbf{Disponibilidade}
\\

A disponibilidade dos alunos é conforme apresentada na seguinte tabela:

\begin{table}[h]
\centering
\begin{tabular}{|l|c|c|}
\hline
\multirow{2}{*}{} & \multicolumn{2}{c|}{Disponibilidade} \\ \cline{2-3} 
                  & 1ºSemestre        & 2ºSemestre       \\ \hline
Pedro Neves       & 20\%              & 40\%             \\ \hline
Rita Capela       & 20\%              & 28,6\%           \\ \hline
Tiago Maurício    & 20\%              & 28,6\%           \\ \hline
Francisco Pires   & 20\%              & 33,3\%           \\ \hline
Alexandre Machado & 20\%              & 28,6\%           \\ \hline
Nuno Silva        & 10\%              & *                \\ \hline
\end{tabular}
\caption{Tabela de Disponibilidade}
\label{disponibilidade}
\end{table}

Para esta tabela foram previstas todas as cadeiras que compõem o projecto, separadas por dois semestres. 
\clearpage

\noindent{\textbf{Organização da equipa}}
\\
\\
A organização dos membros envolvidos vai ser feita em três grupos.
Um grupo para PTR, um para PTI, e um ultimo grupo para os \textit {"elementos moveis"}. 
Estes alunos vão contribuir em conjunto para o trabalho de ambas as cadeiras, e ao mesmo tempo, gerir o funcionamento e as decisões dos grupos.
A decisão de fazer três grupos surgiu em resposta a ter dois membros com competências equivalentes em PTI e PTR, e o acrescento de serem mais competentes em matérias de gestão.

\begin{itemize}
\item Grupo PTR
\begin{itemize}
	\item Francisco Pires
	\item Nuno Silva
\end{itemize}
\item Grupo PTI
\begin{itemize}
	\item Tiago Maurício
	\item Rita Capela
\end{itemize}
\item \textit{Elementos Moveis}
\begin{itemize}
	\item Alexandre Machado
	\item Pedro Neves
\end{itemize}
\end{itemize}

\noindent{\textbf{Tabela de Competencias}}

\begin{table}[h]
\centering
\begin{tabular}{|l|c|c|c|c|c|c|c|}
\hline
                  & PHP & Java & HTML & CSS & Python & Interface & Gestão \\ \hline
Pedro Neves       & 3   & 4    & 3    & 2   & 4      & 1         & 4      \\ \hline
Rita Capela       & 3   & 2    & 4    & 4   & 3      & 4         & 4      \\ \hline
Tiago Maurício    & 3   & 4    & 4    & 3   & 4      & 2         & 3      \\ \hline
Francisco Pires   & 2   & 3    & 4    & 3   & 4      & 3         & 3      \\ \hline
Alexandre Machado & 2   & 4    & 4    & 4   & 4      & 3         & 4      \\ \hline
Nuno Silva        & 2   & 4    & 4    & 4   & 4      & 3         & 3      \\ \hline
\end{tabular}
\caption{Tabela de Competências}
\label{competencias}
\end{table}

\clearpage

\section{Estimação}

Dados históricos

\begin{table}[h]
\centering
\begin{tabular}{|l|l|l|l|l|l|}
\hline
                  & AD              & ASW             & ITW           & ADS           & SO            \\ \hline
Alexandre Machado & 1002 LOC / 160h & 2576 LOC / 160h & 756 LOC / 72h & 454 LOC / 18h & 560 LOC / 42h \\ \hline
Francisco Pires   & 1002 LOC / 160h & NA              & 687 LOC / 5h  & NA            &               \\ \hline
Nuno Silva        &                 &                 &               &               &               \\ \hline
\end{tabular}
\caption{My caption}
\label{my-label}
\end{table}

\subsection{Esforço disponível}

\subsection{Linhas de código}

bla bla bla bla bla bla bla bla bla bla bla bla yo mama bla bla bla bla bla bla bla bla bla bla bla bla bla bla bla bla bla bla bla bla bla bla bla bla bla bla bla bla bla bla bla bla bla bla bla bla bla bla bla 

\begin{table}[h]
\centering
\begin{tabular}{|l|c|c|c|c|}
\hline
                       		& Optimista & Provável & Pessimista & Final \\ \hline
Criar a Base de Dados  		& 50        & 120      & 200        & 123   \\ \hline
Configurar HTTP Server 		& 5         & 20       & 50         & 25    \\ \hline
Ligação à Base de Dados*	& 5         & 10       & 20         & 12    \\ \hline
Segurança              		& 200       & 300      & 350        & 283   \\ \hline
Sistema Distribuído         & 2000      & 3750     & 5000       & 3583   \\ \hline
\textit{Views}            	& 1000      & 1500     & 2500       & 1600  \\ \hline
Controlador                 & 500       & 750      & 1000       & 750   \\ \hline
Modelo                      & 200       & 300      & 500        & 333   \\ \hline
\textbf{Total}		   		& 3210      & 5750     & 8520       & 5827  \\ \hline
\end{tabular}
\caption{Linhas de Código}
\label{codigo}
\end{table}

\subsection{Modelos Empíricos}

\section{Processo de Desenvolvimento de Software}

\section{Planeamento do Projecto}

\section{Gestão de Riscos}

\chapter{Conclusão}

Perante o projecto que nos foi proposto, definimos os requisitos funcionais e não funcionais como pilares da nossa proposta de trabalho. 
Através de uma pesquisa ao \textit {website} do Portal do Utente e um conjunto de boas práticas de serviços \textit {web}, adicionamos funcionalidades possíveis de implementar no SI, e que determinam uma melhoria, tanto no serviço, como na interacção com o utilizador.

\chapter{Bibliografia}

%\begin{thebiblography} {99}

\end{document}