%formatacao do documento e tipo do documento
\documentclass[12pt, a4paper, twoside]{report} %inicio do doc

%pacotes de extensoes
\usepackage[portuges]{babel} %pkg da lingua portugues
\usepackage[latin1,utf8]{inputenc} %pkg da lingua portugues
\usepackage{verbatim} %pkg para escrever sem formataçao
\usepackage{color} %usar cores nas letras
\usepackage{graphicx} %usar imagens no doc
\usepackage[table,xcdraw]{xcolor}
\usepackage{makeidx}
\usepackage{anysize} % para formatar o tamanho do documento


\marginsize{3.17cm}{3.17cm}{2.54cm}{2.54cm}

%fazer indice
\makeindex

\begin{document}

\title{%
	\textbf{Planeamento e Gestão de Projecto}\\ 
	\large Relatório Fase 2
}

\author{%
Alexandre Machado, nº 43551 \\
Nuno Silva, nº 44285 \\
Francisco Pires, nº 44314 \\
}

\date{\today}
\maketitle
\tableofcontents

\chapter{Introdução}

Este projecto tem como objectivo o desenvolvimento e a implementação de um Sistema de Informação (SI), dirigido aos utentes do Serviço Nacional de Saúde (SNS). Este SI é baseado nas tecnologias Web e pretende melhorar a qualidade dos serviços prestados ao utilizador. Após a consulta do Portal da Saúde, e a identificação das capacidades existentes, propomos ampliar os requisitos funcionais disponíveis para o utilizador e melhorar os requisitos não funcionais. Para isso, pretendemos assegurar a melhor disponibilidade, correcção e confidencialidade dos dados registados.

\chapter{Análise de requisitos}

\section{Requisitos funcionais e não funcionais}

\subsection{Requisitos funcionais}

De acordo com os objectivos definidos para este projecto, seleccionamos as seguintes funcionalidades que o utilizador terá disponíveis neste SI.

\begin{itemize}
\item Registo de contactos e dados pessoais
\item Definir agregado familiar 
\item Identificação de cuidador familiar
\item Registo de informação pessoal relevante
\item Registo de indicadores básicos de saúde
\item Registo de exames complementares de diagnóstico
\item Consulta de registos clínicos
\item Pedido de prescrição de medicação crónica
\item Marcação de consultas
\item Inscrição e consulta das listas para cirurgia (eSIGIC)
\item Testamento vital
\item Definir estado no Registo Nacional de Não Dadores (RENNDA)
\item Pesquisa de serviços médicos (directório)
\item Pedido de mudança de médico de família
\item Pedido de isenção de taxas moderadoras

\end{itemize}

\subsection{Requisitos não funcionais}

Para execução das funcionalidades neste SI, será necessário assegurar os requisitos não funcionais que listamos de seguida.

\begin{itemize}
\item Confidencialidade dos dados
\item Segurança dos dados e dos acessos
\item Garantia de disponibilidade
\item Escalável e modular
\item Tempo de resposta
\item Assegurar o cumprimentos das normas legais
\item Resolução de conflitos
\item Persistência e sincronização dos dados
\item Notificações e alertas de acontecimentos do utilizador
\item \textit {Responsive Web Design}
\end{itemize}

\section{Modelo de Casos de Uso}

\section{Esboços de Interfaces}

\section{Modelo de Dados e Requisitos Detalhados}

\chapter{Planeamento}

\section{Recursos}

\section{Estimação}

Aqui esta um exemplo de uma tabela.

\begin{table}[h]
\centering
\begin{tabular}{|l|l|l|l|}
\hline
%\rowcolor[HTML]{9B9B9B}
{Nome} & {Descrição} & {Disponibilidade} (1ºSemestre) & {Moto X}\\ \hline
{Alexandre Machado} & Dual Core 1.2 & Quad Core 1.4 & Hexa Core 1.8\\ \hline
{RAM} & 512MB & 1GB & 3GB\\ \hline
{ROM} & 4G & 8GB/16GB & 32GB/64GB\\ \hline
\end{tabular} \\
\caption{Modelos de smartphones.}
\end{table}

\begin{table}[h]
\centering
\begin{tabular}{|l|l|l|l|}
\hline
%\rowcolor[HTML]{9B9B9B}
& {Linhas} & {Moto G} & {Moto X}\\ \hline
{Criar a Base de Dados} & 1000 & Quad Core 1.4 & Hexa Core 1.8\\ \hline
{Configurar HTTP Server} & 512MB & 1GB & 3GB\\ \hline
{Sistema Distribuido} & 4G & 8GB/16GB & 32GB/64GB\\ \hline
{Segurança} & stuff & stuff & stuff \\ \hline
{Views} & stuff & stuff & stuff \\ \hline
{Controlador} & stuff & stuff & stuff \\ \hline
{Modelo} & stuff & stuff & stuff \\ \hline
\end{tabular} \\
\caption{Decomposição Projecto}
\end{table}

\subsection{Esforço disponível}

\subsection{Linhas de código}

\subsection{Modelos Empíricos}

\section{Processo de Desenvolvimento de Software}

\section{Planeamento do Projecto}

\section{Gestão de Riscos}

\chapter{Arquitectura}

\section{Arquitectura do Sistema}

\chapter{Conclusão}

Perante o projecto que nos foi proposto, definimos os Requisitos Funcionais e Não Funcionais como pilares da nossa proposta de trabalho. Através de uma pesquisa do Website do Portal do Utente e um conjunto de boas práticas de serviços Web, adicionamos funcionalidades possíveis de implementar no SI, e que determinam uma melhoria, tanto no serviço, como na interacção com o utilizador.

\chapter{Bibliografia}

\begin{thebiblography} {99}

\bibitem 

\end{document}